\documentclass[10pt]{article}
\usepackage[left=2.5cm, right=2.5cm, top=2cm, bottom=2.5cm]{geometry}
\usepackage{cmbright}
\usepackage{blindtext}
\usepackage{longtable}
\usepackage{graphicx}
\usepackage{multicol}
\usepackage{caption}
\usepackage{parskip}
\usepackage{subcaption}
\usepackage{siunitx}
\usepackage{lscape}
\usepackage[strict]{changepage}

\title{eDNA Expeditions biodiversity survey of Everglades National Park}

\author{Saara Suominen, Pieter Provoost, Ward Appeltans}
\begin{document}

%\changepage{}{}{}{}{}{-20mm}{}{}{}
%\begin{center}
%  \makebox[\textwidth]{\includegraphics[width=\paperwidth]{images/23.jpg}}
%\end{center}%\maketitle

\maketitle

\section*{Introduction}

\begin{multicols}{2}
Environmental DNA Expeditions is a global, citizen science initiative that will help measure marine biodiversity, and the impacts climate change might have on the distribution patterns of marine life, across UNESCO World Heritage marine sites.

Ocean species shed DNA into the water around them. The genetic material from waste, mucus or cells in one liter of water can determine the species richness in a given area, without the need to actually extract organisms from their environment.
The cost effective, ethical nature of eDNA sampling has the potential to revolutionize knowledge about ecosystems and species diversity and to inspire the next generation of ocean researchers.

eDNA sampling was conducted in Everglades National Park in April 2023, by filtering up to \SI{1800}{\milli\litre} of seawater through filter cartridges with a \SI{0.8}{\micro\metre} pore size. 20 samples were collected at five locations in the park: Trout Creek, North Eagle Creek, Crocodile Dragover, Whip Ray, and Captain. From these 20 samples, a first batch of 11 samples covering all sites has been processed. DNA from these samples was extracted and amplified, and then sent to the sequencing facility.
\end{multicols}

\begin{figure}[h]
\centering
\includegraphics[width=\textwidth]{map}
\caption{Map of eDNA sampling locations}
\end{figure}

\section*{Results}
\subsection*{Sampling and DNA sequencing}

\begin{multicols}{2}
Sequencing of the DNA from 11 samples resulted in over 20 million sequence reads. From these reads we were able to collect 12,507 unique sequences or ASVs. Matching those sequences with reference sequences in public databases, we were able to detect 194 species. A total of 2566 are known from Everglades National Park in the OBIS database. Of the 193 species detected, 77 are not known from the park in OBIS.
\end{multicols}

% latex table generated in R 4.2.2 by xtable 1.8-4 package
% Mon Nov 27 19:00:46 2023
\begin{longtable}{rrr}
  \hline
reads & species & asvs \\ 
  \hline
20543329 & 163 & 12458 \\ 
   \hline
\hline
\end{longtable}


% latex table generated in R 4.2.2 by xtable 1.8-4 package
% Mon Nov 27 11:21:22 2023
\begin{longtable}{llrrrr}
  \hline
locality & materialSampleID & reads & species & asvs & sampleSize \\ 
  \hline
Captain & EE0386 & 1715884 &  33 & 1261 & 1500 \\ 
  Captain & EE0391 & 1504564 &  77 & 5480 & 1800 \\ 
  Captain & EE0402 & 1768280 &  73 & 5873 & 1800 \\ 
  Crocodile Dragover & EE0389 & 1542386 &  64 & 5174 & 1500 \\ 
  Crocodile Dragover & EE0390 & 2070296 &  67 & 5042 & 450 \\ 
  North Eagle Key & EE0388 & 1586226 &  43 & 2831 & 1500 \\ 
  Trout Creek & EE0403 & 1854506 &  86 & 3590 & 1500 \\ 
  Whip Ray & EE0387 & 2902938 &  78 & 3101 & 1800 \\ 
  Whip Ray & EE0404 & 2195910 &  97 & 5838 & 1800 \\ 
  Whip Ray & EE0405 & 1994942 &  97 & 4808 & 1800 \\ 
  Whip Ray & EE0406 & 1491927 & 103 & 4685 & 1800 \\ 
   \hline
\hline
\end{longtable}


\clearpage

\begin{center}
  \makebox[\textwidth]{\includegraphics[width=0.8\paperwidth]{images/23.jpg}}
  \makebox[\textwidth]{\includegraphics[width=0.8\paperwidth]{images/44.jpg}}
  \makebox[\textwidth]{\includegraphics[width=0.8\paperwidth]{images/49.jpg}}
  \makebox[\textwidth]{\includegraphics[width=0.8\paperwidth]{images/54.jpg}}
\end{center}

\clearpage

\begin{figure}[h!]
\centering
\includegraphics[width=\textwidth]{tree}
\caption{Number of DNA reads and species detected by family}
\end{figure}

\begin{multicols}{2}
\blindtext
\blindtext
\end{multicols}

\clearpage

\subsection*{Species detection}

\begin{figure}[h]
     \centering
     \begin{subfigure}[b]{0.45\textwidth}
         \centering
         \includegraphics[width=\textwidth]{images/sphyrna_tiburo.jpg}
         \caption{Sphyrna tiburo}
     \end{subfigure}
     \hfill
     \begin{subfigure}[b]{0.45\textwidth}
         \centering
         \includegraphics[width=\textwidth]{images/hypanus_rudis.jpg}
         \caption{Hypanus rudis}
     \end{subfigure}
     \hfill
     \begin{subfigure}[b]{0.45\textwidth}
         \centering
         \includegraphics[width=\textwidth]{images/epinephelus_itajara.jpg}
         \caption{Epinephelus itajara}
     \end{subfigure}
     \caption{Some of the detected fish species.}
\end{figure}

\clearpage

\section*{Appendix: species list}

\begin{landscape}
% latex table generated in R 4.2.2 by xtable 1.8-4 package
% Mon Nov 27 14:29:03 2023
\begingroup\fontsize{9pt}{10pt}\selectfont
\begin{longtable}{lllll}
  \hline
phylum & class & species & category & new \\ 
  \hline
Amoebozoa & Discosea & Paramoeba aestuarina &  & yes \\ 
  Amoebozoa & Myxogastrea & Craterium leucocephalum &  & yes \\ 
  Annelida & Clitellata & Thalassodrilides gurwitschi &  & yes \\ 
  Annelida & Polychaeta & Marphysa sanguinea &  &  \\ 
  Annelida & Polychaeta & Bhawania goodei &  & yes \\ 
  Annelida & Polychaeta & Glycinde multidens &  &  \\ 
  Annelida & Polychaeta & Polydora websteri &  & yes \\ 
  Annelida & Polychaeta & Prionospio steenstrupi &  &  \\ 
  Annelida & Polychaeta & Timarete caribous &  & yes \\ 
  Annelida & Polychaeta & Pectinaria gouldii &  &  \\ 
  Annelida & Polychaeta & Loimia medusa &  &  \\ 
  Annelida &  & Golfingia elongata &  & yes \\ 
  Annelida &  & Siphonosoma cumanense &  & yes \\ 
  Arthropoda & Copepoda & Acartia tonsa &  & yes \\ 
  Arthropoda & Hexapoda & Rhopalosiphum padi &  & yes \\ 
  Arthropoda & Hexapoda & Macrosteles quadrilineatus &  & yes \\ 
  Arthropoda & Malacostraca & Cymadusa compta &  &  \\ 
  Arthropoda & Malacostraca & Elasmopus nkjaf &  & yes \\ 
  Arthropoda & Malacostraca & Neopanope packardii &  &  \\ 
  Arthropoda & Malacostraca & Callinectes sapidus &  &  \\ 
  Arthropoda & Malacostraca & Portunus sanguinolentus &  & yes \\ 
  Arthropoda & Malacostraca & Paracerceis caudata &  &  \\ 
  Arthropoda & Merostomata & Limulus polyphemus & VU &  \\ 
  Arthropoda & Pycnogonida & Callipallene brevirostris &  &  \\ 
  Ascomycota & Dothideomycetes & Cladosporium herbarum &  & yes \\ 
  Ascomycota & Eurotiomycetes & Aspergillus versicolor &  & yes \\ 
  Bacillariophyta & Bacillariophyceae & Cylindrotheca closterium &  & yes \\ 
  Bacillariophyta & Bacillariophyceae & Haslea crucigera &  & yes \\ 
  Bacillariophyta & Bacillariophyceae & Navicula minima &  & yes \\ 
  Bacillariophyta & Bacillariophyceae & Phaeodactylum tricornutum &  & yes \\ 
  Bryozoa & Gymnolaemata & Bugula rochae &  & yes \\ 
  Bryozoa & Gymnolaemata & Amathia evelinae &  & yes \\ 
  Cercozoa & Chlorarachnea & Chlorarachnion reptans &  & yes \\ 
  Chlorophyta & Mamiellophyceae & Dolichomastix tenuilepis &  & yes \\ 
  Chlorophyta & Mamiellophyceae & Micromonas pusilla &  &  \\ 
  Chordata & Ascidiacea & Ecteinascidia styeloides &  & yes \\ 
  Chordata & Ascidiacea & Botrylloides niger &  &  \\ 
  Chordata & Ascidiacea & Botrylloides simodensis &  & yes \\ 
  Chordata & Aves & Anas poecilorhyncha &  & yes \\ 
  Chordata & Aves & Ardea cinerea &  & yes \\ 
  Chordata & Aves & Falco sparverius &  & yes \\ 
  Chordata & Elasmobranchii & Negaprion brevirostris & VU &  \\ 
  Chordata & Elasmobranchii & Sphyrna tiburo & EN &  \\ 
  Chordata & Elasmobranchii & Hypanus americanus & NT &  \\ 
  Chordata & Elasmobranchii & Hypanus rudis & CR & yes \\ 
  Chordata & Mammalia & Mirounga leonina &  & yes \\ 
  Chordata & Mammalia & Tursiops truncatus &  &  \\ 
  Chordata & Mammalia & Neophocaena phocaenoides & VU & yes \\ 
  Chordata & Mammalia & Trichechus manatus & VU &  \\ 
  Chordata & Teleostei & Chaetodipterus faber &  &  \\ 
  Chordata & Teleostei & Albula glossodonta & VU & yes \\ 
  Chordata & Teleostei & Albula vulpes & NT &  \\ 
  Chordata & Teleostei & Myrophis punctatus &  &  \\ 
  Chordata & Teleostei & Myrophis vafer &  & yes \\ 
  Chordata & Teleostei & Atherinomorus stipes &  &  \\ 
  Chordata & Teleostei & Opsanus beta &  &  \\ 
  Chordata & Teleostei & Opsanus tau &  &  \\ 
  Chordata & Teleostei & Strongylura notata &  &  \\ 
  Chordata & Teleostei & Strongylura timucu &  &  \\ 
  Chordata & Teleostei & Tylosurus crocodilus &  &  \\ 
  Chordata & Teleostei & Chriodorus atherinoides &  &  \\ 
  Chordata & Teleostei & Chasmodes bosquianus &  &  \\ 
  Chordata & Teleostei & Centropomus undecimalis &  &  \\ 
  Chordata & Teleostei & Sphyraena barracuda &  &  \\ 
  Chordata & Teleostei & Sphyraena jello &  & yes \\ 
  Chordata & Teleostei & Caranx crysos &  &  \\ 
  Chordata & Teleostei & Caranx hippos &  &  \\ 
  Chordata & Teleostei & Caranx ignobilis &  & yes \\ 
  Chordata & Teleostei & Caranx latus &  &  \\ 
  Chordata & Teleostei & Oligoplites saurus &  &  \\ 
  Chordata & Teleostei & Selene vomer &  &  \\ 
  Chordata & Teleostei & Trachinotus carolinus &  &  \\ 
  Chordata & Teleostei & Echeneis naucrates &  &  \\ 
  Chordata & Teleostei & Brevoortia tyrannus &  & yes \\ 
  Chordata & Teleostei & Harengula humeralis &  & yes \\ 
  Chordata & Teleostei & Harengula jaguana &  &  \\ 
  Chordata & Teleostei & Opisthonema oglinum &  &  \\ 
  Chordata & Teleostei & Anchoa mitchilli &  &  \\ 
  Chordata & Teleostei & Jenkinsia lamprotaenia &  &  \\ 
  Chordata & Teleostei & Carassius auratus &  & yes \\ 
  Chordata & Teleostei & Cyprinodon variegatus &  &  \\ 
  Chordata & Teleostei & Floridichthys carpio &  &  \\ 
  Chordata & Teleostei & Adinia xenica &  & yes \\ 
  Chordata & Teleostei & Fundulus confluentus &  &  \\ 
  Chordata & Teleostei & Fundulus grandis &  &  \\ 
  Chordata & Teleostei & Lucania goodei &  & yes \\ 
  Chordata & Teleostei & Lucania parva &  &  \\ 
  Chordata & Teleostei & Belonesox belizanus &  &  \\ 
  Chordata & Teleostei & Gambusia affinis &  &  \\ 
  Chordata & Teleostei & Gambusia holbrooki &  &  \\ 
  Chordata & Teleostei & Gambusia rhizophorae &  &  \\ 
  Chordata & Teleostei & Poecilia latipinna &  &  \\ 
  Chordata & Teleostei & Poecilia sphenops &  & yes \\ 
  Chordata & Teleostei & Elops saurus &  &  \\ 
  Chordata & Teleostei & Megalops atlanticus & VU &  \\ 
  Chordata & Teleostei & Diapterus auratus &  &  \\ 
  Chordata & Teleostei & Eucinostomus argenteus &  &  \\ 
  Chordata & Teleostei & Eucinostomus gula &  &  \\ 
  Chordata & Teleostei & Eucinostomus jonesii &  &  \\ 
  Chordata & Teleostei & Eucinostomus melanopterus &  &  \\ 
  Chordata & Teleostei & Eugerres plumieri &  &  \\ 
  Chordata & Teleostei & Gerres cinereus &  &  \\ 
  Chordata & Teleostei & Ulaema lefroyi &  &  \\ 
  Chordata & Teleostei & Haemulon aurolineatum &  &  \\ 
  Chordata & Teleostei & Haemulon parra &  &  \\ 
  Chordata & Teleostei & Haemulon plumierii &  &  \\ 
  Chordata & Teleostei & Haemulon sciurus &  &  \\ 
  Chordata & Teleostei & Lutjanus argentimaculatus &  & yes \\ 
  Chordata & Teleostei & Lutjanus griseus &  &  \\ 
  Chordata & Teleostei & Nicholsina usta &  &  \\ 
  Chordata & Teleostei & Cynoscion nebulosus &  &  \\ 
  Chordata & Teleostei & Pogonias cromis &  &  \\ 
  Chordata & Teleostei & Archosargus probatocephalus &  &  \\ 
  Chordata & Teleostei & Archosargus rhomboidalis &  &  \\ 
  Chordata & Teleostei & Diplodus kotschyi &  & yes \\ 
  Chordata & Teleostei & Lagodon rhomboides &  &  \\ 
  Chordata & Teleostei & Gobiesox strumosus &  &  \\ 
  Chordata & Teleostei & Gobiosoma bosc &  &  \\ 
  Chordata & Teleostei & Gobiosoma ginsburgi &  & yes \\ 
  Chordata & Teleostei & Gobiosoma robustum &  &  \\ 
  Chordata & Teleostei & Lophogobius cyprinoides &  &  \\ 
  Chordata & Teleostei & Microgobius gulosus &  &  \\ 
  Chordata & Teleostei & Microgobius microlepis &  &  \\ 
  Chordata & Teleostei & Mugil cephalus &  &  \\ 
  Chordata & Teleostei & Mugil curema &  &  \\ 
  Chordata & Teleostei & Mugil rubrioculus &  &  \\ 
  Chordata & Teleostei & Ogilbia cayorum &  &  \\ 
  Chordata & Teleostei & Epinephelus itajara & VU &  \\ 
  Chordata & Teleostei & Epinephelus lanceolatus &  & yes \\ 
  Chordata & Teleostei & Trinectes maculatus &  &  \\ 
  Chordata & Teleostei & Ariopsis felis &  &  \\ 
  Chordata & Teleostei & Anarchopterus criniger &  &  \\ 
  Chordata & Teleostei & Hippocampus zosterae &  &  \\ 
  Chordata & Teleostei & Syngnathus floridae &  &  \\ 
  Chordata & Teleostei & Syngnathus fuscus &  &  \\ 
  Chordata & Teleostei & Syngnathus schlegeli &  & yes \\ 
  Chordata & Teleostei & Syngnathus scovelli &  &  \\ 
  Chordata & Teleostei & Chilomycterus schoepfii &  &  \\ 
  Chordata & Teleostei & Stephanolepis cirrhifer &  & yes \\ 
  Chordata & Teleostei & Stephanolepis hispida &  &  \\ 
  Chordata & Teleostei & Sphoeroides maculatus &  &  \\ 
  Chordata & Teleostei & Sphoeroides parvus &  & yes \\ 
  Chordata & Teleostei & Sphoeroides spengleri &  &  \\ 
  Chordata &  & Caretta caretta & VU &  \\ 
  Cnidaria & Anthozoa & Exaiptasia diaphana &  &  \\ 
  Cnidaria & Anthozoa & Boloceroides mcmurrichi &  & yes \\ 
  Cnidaria & Anthozoa & Diadumene leucolena &  & yes \\ 
  Cnidaria & Anthozoa & Edwardsia longicornis &  & yes \\ 
  Cnidaria & Anthozoa & Nanozoanthus harenaceus &  & yes \\ 
  Cnidaria & Hydrozoa & Zancleopsis dichotoma &  & yes \\ 
  Cnidaria & Hydrozoa & Clytia hemisphaerica &  & yes \\ 
  Cnidaria & Hydrozoa & Obelia bidentata &  & yes \\ 
  Ctenophora & Tentaculata & Vallicula multiformis &  & yes \\ 
  Echinodermata & Holothuroidea & Leptosynapta tenuis &  &  \\ 
  Echinodermata & Holothuroidea & Thyonella gemmata &  & yes \\ 
  Echinodermata & Holothuroidea & Sclerodactyla briareus &  &  \\ 
  Echinodermata & Ophiuroidea & Amphipholis squamata &  &  \\ 
  Gnathostomulida &  & Gnathostomula axi &  & yes \\ 
  Mollusca & Bivalvia & Arcopsis adamsi &  &  \\ 
  Mollusca & Bivalvia & Ameritella mitchelli &  & yes \\ 
  Mollusca & Bivalvia & Brachidontes exustus &  &  \\ 
  Mollusca & Bivalvia & Crassostrea virginica &  &  \\ 
  Mollusca & Bivalvia & Anomalocardia auberiana &  & yes \\ 
  Mollusca & Bivalvia & Chione cancellata &  &  \\ 
  Mollusca & Gastropoda & Philinopsis pusa &  & yes \\ 
  Mollusca & Gastropoda & Bulla arabica &  & yes \\ 
  Mollusca & Gastropoda & Acteocina canaliculata &  &  \\ 
  Mollusca & Gastropoda & Crepidula convexa &  &  \\ 
  Mollusca & Gastropoda & Learchis poica &  & yes \\ 
  Mollusca & Gastropoda & Nanuca occidentalis &  & yes \\ 
  Mollusca & Gastropoda & Bittiolum varium &  &  \\ 
  Mollusca & Gastropoda & Cerithium eburneum &  &  \\ 
  Mollusca & Gastropoda & Cerithium muscarum &  &  \\ 
  Mollusca & Gastropoda & Modulus modulus &  &  \\ 
  Mollusca & Gastropoda & Cylindrobulla beauii &  &  \\ 
  Mollusca & Gastropoda & Cyerce antillensis &  & yes \\ 
  Myzozoa & Dinophyceae & Amphidinium massartii &  & yes \\ 
  Nemertea & Hoplonemertea & Tetrastemma elegans &  & yes \\ 
  Nemertea & Hoplonemertea & Tetrastemma wilsoni &  & yes \\ 
  Nemertea & Palaeonemertea & Tubulanus riceae &  & yes \\ 
  Phoronida &  & Phoronis psammophila &  & yes \\ 
  Porifera & Demospongiae & Callyspongia diffusa &  & yes \\ 
  Porifera & Demospongiae & Callyspongia vaginalis &  & yes \\ 
  Porifera & Demospongiae & Cladocroce burapha &  & yes \\ 
  Porifera & Demospongiae & Haliclona vansoesti &  & yes \\ 
  Porifera & Demospongiae & Gelliodes wilsoni &  & yes \\ 
  Porifera & Demospongiae & Pachychalina tenera &  & yes \\ 
  Porifera & Demospongiae & Mycale fibrexilis &  & yes \\ 
  Porifera & Demospongiae & Tedania ignis &  & yes \\ 
  Porifera & Demospongiae & Amorphinopsis fenestrata &  & yes \\ 
  Porifera & Demospongiae & Halichondria melanadocia &  & yes \\ 
  Porifera & Demospongiae & Geodia neptuni &  & yes \\ 
  Porifera & Demospongiae & Poecillastra laminaris &  & yes \\ 
  Rhodophyta & Florideophyceae & Polysiphonia echinata &  & yes \\ 
   \hline
\hline
\end{longtable}
\endgroup

\end{landscape}

\end{document}
